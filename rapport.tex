\documentclass{article}      % Specifies the document class

% -------------------- Packages --------------------
\usepackage{amsmath}
\usepackage{amssymb}
\usepackage[noend]{algpseudocode}
\usepackage{algorithm}
\usepackage{graphicx}
\usepackage{float}
\usepackage{fontawesome5}
\usepackage{listings}

\lstset{language=Python,keywordstyle={\bfseries \color{blue}}}
\NewDocumentCommand{\codeword}{v}{%
    \texttt{\textcolor[HTML]{5c5c65}{#1}}%
}


\usepackage{hyperref}
\hypersetup{
    colorlinks=true,
    linkcolor=blue,
    filecolor=magenta,      
    urlcolor=cyan,
    pdftitle={Overleaf Example},
    pdfpagemode=FullScreen,
    }

\urlstyle{same}

\usepackage{bookmark}
\hypersetup{hidelinks} %enlève les cadres rouges autour des hyperliens


% ---------- PSEUDO CODE : hack to remove indent ----------
% https://tex.stackexchange.com/questions/354564/how-to-remove-leading-indentation-from-algorithm
\usepackage{xpatch}
\makeatletter
\xpatchcmd{\algorithmic}
  {\ALG@tlm\z@}{\leftmargin\z@\ALG@tlm\z@}
  {}{}
\makeatother

\usepackage{xcolor}
\usepackage[framemethod=tikz]{mdframed}
\usepackage{tikzpagenodes}
\usetikzlibrary{calc}

% add foreach
\algnewcommand\algorithmicforeach{\textbf{for each}}
\algdef{S}[FOR]{ForEach}[1]{\algorithmicforeach\ #1\ \algorithmicdo}



% -------------------- Couleurs --------------------
\definecolor{definition}{HTML}{2f80ed}
\definecolor{definition-bg}{HTML}{e0ecfd}

\definecolor{danger}{HTML}{e6505f}
\definecolor{danger-bg}{HTML}{fce5e7}

\definecolor{exogris}{gray}{0.4}



% -------------------- Code --------------------
\definecolor{codegreen}{rgb}{0,0.6,0}
\definecolor{codegray}{rgb}{0.5,0.5,0.5}
\definecolor{codepurple}{rgb}{0.58,0,0.82}
\definecolor{backcolour}{rgb}{0.95,0.95,0.92}

\lstdefinestyle{code-style}{
    backgroundcolor=\color{backcolour},   
    commentstyle=\color{codegreen},
    keywordstyle=\color{magenta},
    numberstyle=\tiny\color{codegray},
    stringstyle=\color{codepurple},
    basicstyle=\ttfamily\footnotesize,
    breakatwhitespace=false,         
    breaklines=true,                 
    captionpos=b,                    
    keepspaces=true,                 
    numbers=left,                    
    numbersep=5pt,
    showspaces=false,                
    showstringspaces=false,
    showtabs=false,                  
    tabsize=2
}

% -------------------- Styles --------------------
\mdfdefinestyle{definition-style}{%
  innertopmargin=10px,
  innerbottommargin=10px,
  linecolor=definition,
  backgroundcolor=definition-bg,
  roundcorner=4px
}
\newmdenv[style=definition-style]{definition}

\mdfdefinestyle{danger-style}{%
  innertopmargin=10px,
  innerbottommargin=10px,
  linecolor=danger,
  backgroundcolor=danger-bg,
  roundcorner=4px
}
\newmdenv[style=danger-style]{danger}


% -------------------- Document --------------------
\title{Probabilités et Statistiques\\\Large{Projet noté}}
\author{MADANI Abdenour\\TRIOLET Hugo}
\date{Licence 3\\2021 - 2022}
\begin{document}
\normalsize
\maketitle

\renewcommand*\contentsname{Table des matières}
\tableofcontents
\newpage



\section{Introduction}
\subsection{Objectifs}
Les objectifs de ce TPs sont :
\begin{itemize}
  \item Orienter un graphe non-orienté en un graphe fortement connexe,
  \item Décomposer en graphe en chaînes (selon Schmidt$^\text{\cite{schmidt}}$),
  \item Établir si un graphe est 2-connexe ou 2-arête,
  \item Calculer les composantes 2-connexes et 2-arêtes-connexes sinon.
\end{itemize}

On utilisera pour ceci \textbf{SageMath} (bibliothèque de fonctions pour Python).



\subsection{Définitions}
\begin{definition}
{ \scriptsize \textcolor{definition}{\faIcon{graduation-cap} \textbf{DÉFINITION}}}
\vspace{3px}
\\ \underline{\textbf{Pont}}
\vspace{2.5px}
\\ Arête dont la suppression augmente le nombre de composantes connexes du graphe restant.%
\\ \textit{(aussi appelé arête déconnectante)}
\end{definition}


\subsection{Résumé de notre approche}
On implémente la décomposition en chaînes$^\text{\cite{schmidt}}$.
\\Celle-ci nous permet d'obtenir tous les ponts, ainsi que tous les sommets d'articulation du graphe.
\\On en déduit ensuite si le graphe est 2-connexe, 2-arête-connexe, ou aucun des deux, grâce à l'Algorithme 1 de Schmidt$^\text{\cite{schmidt}}$.
\\On calcule ensuite les composantes 2-connexes et 2-arêtes-connexes.
\\\\Vis-à-vis du code, nous l'avons documenté à l'aide de la \textit{docstring} de Python, les fonctions se comprennent donc naturellement grâce à celle-ci.
%


\section{Exercices}
\subsection{Exercice 1}
\textit{\textcolor{exogris}{
Dans une premier temps il est demandé d’implémenter les algorithmes de calcul de 2-connexité dû à Schmidt$^\text{\cite{schmidt}}$.
\\(1) Calculer les composantes 2-connexes d’un graphe.
\\(2) Calculer les composantes 2-arêtes-connexes d’un graphe.
}}
\\On commence par effectuer la décomposition en chaînes.
\\\\Grâce aux Lemmes 4 et 5$^\text{\cite{schmidt}}$, on peut identifier les ponts et sommets d'articulation du graphe.
\\\\\textbf{Pour obtenir les composantes 2-connexes :}
\\On prend le graphe original, on le copie.
\\Ensuite, on fait un traitement différent sur chaque pont, puis sur chaque sommet d'articulation en début de cycle.


% ---------- Composantes 2-sommet-connexe  ---------- 
\begin{algorithm}[H] \caption{Composantes 2-sommets-connexes}
\begin{algorithmic}

\\ \State \textbf{Entrée :} $ponts$, la liste des ponts, ainsi que $chaines$ la liste des chaînes, en variable globale
\State \textbf{Sortie :} Les composantes 2-sommets-connexes

\\ \State Prenons une copie du graphe du départ, dans laquelle nous allons effectuer nos modifications.

\\ \ForEach {pont $(u, v)$}
    \State On supprime l'arête $(u, v)$ du graphe.
    \State On rajoute un noeud $u'$ (si déjà pris, $u_2$, puis $u_3$, etc) et de même un noeud $v'$.
    \State On rajoute une arête $(u', v')$ dans le graphe.
\EndFor

\\ \ForEach {sommet $u$ en début de cycle, à partir de $C_2$}
    \State Le cycle est de la forme : $u, v_1, ..., v_k, u$
    \\\State On supprime les arêtes $(u, v_1)$ et $(u, v_k)$ du graphe.        
    \State On rajoute un noeud $u'$ (si déjà pris, $u_2$, puis $u_3$, etc)
    \State On rajoute deux arêts $(u', v_1)$ et $(u', v_k)$ dans le graphe.
    
    \\\State \textbf{SAUF :} 
    \State Si le sommet est de degré 2 (c'est-à-dire que ses autres arêtes ont été supprimées entre temps).
    \State Dans ce cas, on ne fait rien.
\EndFor

\\ \State \textbf{return} le graphe des composantes 2-sommets-connexes
\end{algorithmic}
\end{algorithm}
% ---------- END Composantes 2-sommet-connexe  ---------- 


Voir fonction \codeword{calcule_comp_2_sommet_connexe}.
%
%
\\\\\textbf{Pour obtenir les composantes 2-arêtes-connexes :}
\\On prend le graphe original, on supprime ses ponts.
\\Ensuite, on enlève les noeuds restants de degré 0.
\\Voir fonction \codeword{calcule_comp_2_arete_connexe}.
%



\subsection{Exercice 2}
\textit{\textcolor{exogris}{
Trouvez un orientation en un graphe fortement connexe$^\text{\cite{schmidt}}$ ou trouvez une arête déconnectante.
}}
\\On effectue un DFS pour orienter toutes les arêtes, tel que décrit dans la \textit{Figure 1b}$^\text{\cite{schmidt}}$.
\\On oriente les arcs de parenté du fils vers le père, et les arcs arrières du père vers le fils.
\\D'après le théorème de Robbins$^\text{\cite{robbins}}$, le graphe orienté obtenu sera fortement connexe si et seulement si celui-ci était 2-arête-connexe.
\\\\Au sein de notre fonction \codeword{parcours_graphe()}, nous avons une sous-fonction \codeword{parcours()} qui est une fonction récursive de DFS.
\\Si jamais le graphe n'est pas connexe, nous faisons appel à la fonction \codeword{lance_parcours()} qui lance un DFS sur chaque sommet non-visité.
\\Durant la fonction \codeword{parcours()}, nous allons créer un nouveau graphe qui contiendra, en \textcolor[HTML]{333333}{gris foncé}, \textbf{l'arbre de parcours en profondeur}, et en \textcolor[HTML]{a0a0a0}{gris clair} tous les arcs arrières.
\begin{danger}
{ \scriptsize \textcolor{danger}{\faIcon{exclamation-triangle} \textbf{ATTENTION}}}
\vspace{3px}
\\ Ce graphe est appelé \codeword{arbre_parcours} dans le code (bien qu'il contienne les arcs arrière).
\end{danger}

\begin{figure}[H]
    \centering
     \scalebox{.35}{\includegraphics{graphe-oriente-gris.png}}
    \\\codeword{arbre_parcours}
\end{figure}

Pour plus de commodités, nous créons aussi deux graphes à part contenant chacun soit uniquement l'arbre de parcours, soit uniquement les arrières : appelés respectivement \codeword{arbre_parcours_uniquement} et \codeword{arriere}.

\begin{figure}[H]
    \centering
     \scalebox{.5}{\includegraphics{arbre_parcours_uniquement.png}}
    \\\codeword{arbre_parcours_uniquement}
\end{figure}

\begin{figure}[H]
    \centering
     \scalebox{.5}{\includegraphics{arriere.png}}
    \\\codeword{arriere}
\end{figure}

Le graphe \codeword{arbre_parcours} est donc une orientation en graphe fortement connexe si et seulement si la variable \codeword{deux_arete_connexe} renvoyée par la fonction vaut \codeword{True}.
\\Si \codeword{deux_arete_connexe} vaut \codeword{False}, alors il suffit de regarder la variable \codeword{ponts} renvoyée par la fonction (ou \codeword{graphe_ponts.edges()}).



\subsection{Exercice 3}
\textit{\textcolor{exogris}{
Montrez qu’un graphe est fortement connexe$^\text{\cite{schmidt}}$ si et seulement si chaque arc est présent dans au moins un circuit de G.
}}
\\$\implies$ : On suppose que quelque soit $e$ appartenant à E, $e$ est présent dans au moins un circuit de G. On pose $e$ = {x,y}, avec x,y appartenant à V.
\\Puisque $e$ est un chemin du sommet x au sommet et que $e$ est présent dans au moins un circuit de G, cela veut dire que le chemin de y à x existe.
\\Autrement dit, qq soit e un arc tel que e = (x, y) (resp. (y, x)), alors qq soit x, y appartenant à V, il existe un chemin de x à y (resp. y à x) (donc e) et il existe un chemin de y à x (resp. de x à y). Donc, par définition, cela implique que le graphe est fortement connexe.
%
\\\\$\impliedby$ :  Par contraposée, on suppose qu'il existe un arc e appartenant à E, tel que e n'apparaisse dans aucun des circuits de G. 
\\Ainsi, si pose e = (x, y) (resp. (y, x)), avec x, y appartenant à V, alors cela veut dire qu'il existe un chemin de x à y (resp. de y à x) mais il n'en existe pas de y à x (resp. de x à y)
\\Donc le graphe G n'est pas fortement connexe.



\subsection{Exercice 4}
\textit{\textcolor{exogris}{
Prouvez ou infirmez l’affirmation suivante :
\\Un graphe est fortement connexe si et seulement si son graphe sous-jacent est 2-arête connexe.
}}
\\Voici un exemple qui infirme la proposition.
Soit ce graphe non-orienté :

\begin{figure}[H]
    \centering
     \scalebox{.35}{\includegraphics{exo4-g2.png}}
    \\Un graphe orienté non fortement connexe
\end{figure}

Celui-ci n'est pas fortement connexe, car on ne peut pas accéder à 1 à partir de 3 par exemple.
Or voici son graphe sous-jacent :
\begin{figure}[H]
    \centering
     \scalebox{.35}{\includegraphics{exo4-g1.png}}
    \\Son graphe sous-jacent
\end{figure}
Celui-ci est bien 2-arête-connexe, pourtant le graphe original n'était pas fortement connexe, cela conclut la preuve.



\section{Exemples d'utilisation du code}
\subsection{Comment utiliser le code}
Il faut évidemment disposer de SageMath pour exécuter le code.
Par exemple, en l'installant sur le site officiel, et en l'exécutant dans un NoteBook tel que Jupyter.
%
\\ On commence par exécuter le code contenant toutes les fonctions ainsi que quelques variables globales.
%
\\ On crée un graphe, puis on utilise la fonction \codeword{affiche_infos()}

\begin{lstlisting}[style=code-style]
graphe_exemple_schmidt = Graph()
graphe_exemple_schmidt.add_edges([(1, 2), (1, 3), (1, 4), (2, 3), (2, 5), (3, 4), (3, 5), (5, 6), (5, 9), (5, 10), (6, 7), (6, 8), (7, 8), (9, 10)])

infos = affiche_infos(graphe_exemple_schmidt)
\end{lstlisting}

\begin{figure}[H]
    \centering
     \scalebox{0.65}{\includegraphics{exemple-code1.png}}
    \\Exemple d'utilisation 1
\end{figure}

Dans la fonction \codeword{parcours_graphe()}, à la fin, on peut voir un dictionnaire \codeword{informations}.
On peut lui extraire les informations que l'on désire.
%
\\ Pour afficher un arbre de parcours avec les couleurs du rapport, on utilise \codeword{plot_couleur}.

\begin{lstlisting}[style=code-style]
plot_couleur(infos['arbre_parcours'])
\end{lstlisting}
\begin{figure}[H]
    \centering
     \scalebox{0.35}{\includegraphics{graphe-oriente-gris.png}}
\end{figure}

Affichage des composantes 2-connexes avec les sommets d'articulation séparés de la même couleur :
\begin{lstlisting}[style=code-style]
affiche_comp_2_sommet_connexe(graphe_exemple_schmidt)
\end{lstlisting}
\begin{figure}[H]
    \centering
     \scalebox{0.5}{\includegraphics{comp2s-couleur.png}}
\end{figure}

Affichage des composantes 2-arêtes-connexes
\begin{lstlisting}[style=code-style]
plot_couleur(infos['composantes_2_arete_connexe'])
\end{lstlisting}
\begin{figure}[H]
    \centering
     \scalebox{0.5}{\includegraphics{comp2a.png}}
\end{figure}


\section{Références}
\begin{thebibliography}{9}
\bibitem{robbins}
Herbert Robbins, \textit{A theorem on graphs, with an application to a problem on traffic control}, American Mathematical Monthly 46, 281–283.

\bibitem{schmidt}
Jens M. Schmidt, \href{https://arxiv.org/ftp/arxiv/papers/1209/1209.0700.pdf}{\underline{\textit{A Simple Test on 2-Vertex- and 2-Edge-Connectivity}}},
\\Inf. Process. Lett. \textbf{113} (2013), no. 7, 241–244.
\end{thebibliography}

\end{document}
